http://mathb.in/68309
http://mathb.in/68362

https://d3x0r.github.io/STFRPhysics/3d/indexSphereMap3.html
(Hidden link above)




- $t$ is the angle around the sphere (longitude); range is $0$ to $2\pi$ 
- $r$ is the latitude 2pi is the same pole, but different representation; range is $0$ to $4\pi$ ($-2\pi$ to $2\pi$)
- $\delta r$ follows a geodesic that forward parallel transpoarts the tangent plane; (does not change the orientation of the tangent plane moving around this path.
- When theta is 0 ($t=\frac \pi 2$ and $r=\pi$), so at the south pole, at 90 degrees to the right, then the $\cos t$ and $\cos \frac r 2 $ are also 0, and the result is the identity rotation (0,0,0).  Most other points that are at a southern pole are separated by their longitude angle.  
- the poles where $r=0$, $r=2\pi$ are at $(+\pi,0)$ and $(-\pi,0)$  respectively.  A small angle of $r$ moves away from the pole fairly quickly; but while at these poles, it's impossible to determine the $t$.
- n is a combined scalar; $\frac 1 {\sin \frac \theta 2} $ normalizes the vector x and y; $\theta$ is then applied as a scalar on the unit length.
- $xx+yy=\theta\theta$

$ (r,t) => (x,y) $

$\theta = \sin^{-1}( \sin(t)*\sin \frac r 2 )*2 + \pi$

$n = \frac \theta { \sin \frac \theta 2 }$ 

$ x = n \cos \frac r 2 $

$ y = n \cos(t)   \sin {\frac r 2}$
			
---

- reverse function... 

$ (x,y)=> (r,t)$

$ \theta = \sqrt xx+yy $

$ r = 2 \cos^{-1} {\frac {x \sin \frac \theta 2  } \theta  }$

$ t = \cos ^{-1} {\frac { y \sin {\frac \theta 2} } {\theta \sin {\frac r 2 } } }$

---
## 3D transformation

- $r$ is still the distance from pole to pole
- $t$ is still the angle around the pole.
- $s$ is an offset to the base and angle around the pole.
- the value ( $t$-$s$ ) might often be used to maintain the same up orientation and just apply a turn to the tangent plant at the same point.
- $s$ from 0 to pi rotates the angle by 360 degrees, so the resulting twist is the same orientation; but at the opposite same-pole.

$ (r,t,s) => (x,y,z) $

$\theta = \sin^{-1}( \sin(s+t)*\sin \frac r 2 )*2 + \pi$

$n = \frac \theta { \sin \frac \theta 2 }$ 

$ x = n \cos s \cos \frac r 2 $

$ y = n \cos(s+t)   \sin {\frac r 2}$

$ z = n \sin s \cos \frac r 2 $

---

## 4D rotation(?)

- Rotate the up, but don't twist the base frame.... which yields frame torsion.
- as pole configurations go...
- `twistDelta` turns the frame that gets forward parallel transported along R at some `lng`.   following along a constant lng does not rotate the tangent plane relative to the movement.

- gamma // used as angle around the pole delta

- lng  // used as bias angle from (0+gamma) angle

- twistDelta // rotates the tangent plane

- r   // distance from one pole to another - follows a geodesic that forward parallel transpoarts the tangent plane.


						let ang = Math.acos( -Math.sin(gamma+2*lng-twistDelta)*Math.sin(r/2) )*2 ;
						const Cx = Math.cos( +twistDelta)         * Math.cos(r/2);
						const Cy = Math.sin( +twistDelta)         * Math.cos(r/2);
						const Cz = Math.cos(2*lng + gamma-twistDelta) * Math.sin(r/2);
