http://mathb.in/51333

See Also:  mathb.in/45267

Rotate vector V to V'
---
###given
 $V$ is a vector representing a point in space to be rotated around an axis

($A$, $a$) defining the axis and angle to rota- te; this is also a definition of a frame...

###result
 $V'$ is a vector rotated by the scaled frame.
 


---

- $ V' =  \cos(a) V + \sin( a )( V \times A) + (1-\cos( a )) ( A \cdot V ) A $



### notes:
 - To apply inverse multiply $a$ by $-1$.  The axis might also be multiplied by -1; the result is the same, computationally it's fewer operations to just modify the angle.
 - to apply a partial timestep scale $a$ by $dT$ (something like delta-T)


---
## Rotate A around B

###given
A frame defined by ($A$, $a$) as the Axis and Angle, or  $(x,y,z)$ unit vector, and $\theta$. 

Another frame ($B$, $b$) as the Axis and Angle  $(x,y,z)$ unit vector, and $\theta$.

###result 

($R'$,$\theta$) that is the composite frame rotating A around B.

---

- $  \theta = 2 \cos^{-1} ( \cos\frac a 2\cos\frac b 2-\sin\frac a 2\sin\frac b 2 ( A \cdot B ) )  $
  
- $   R = (\sin\frac b 2\cos\frac b 2)A 
           + (\sin\frac a 2\cos\frac b 2)B 
           + (\sin\frac a 2\sin\frac b 2)( B \times A )$

- $  R' = \frac R {||R||} $  

or

- $\theta = 2 \cos^{-1} ( \frac { (1 - A \cdot B) \cos( \frac {b - a} 2) + (1 + A \cdot B) \cos(\frac {b + a} 2) } 2 ) $


      
- $  R = (\sin \frac {b + a} 2 - \sin \frac {b - a} 2) A 
           + ( \sin\frac {b - a} 2 + \sin\frac {b + a} 2) B 
           + ( \cos\frac {b - a} 2 - \cos\frac {b + a} 2)( B \times A )$

  - this version of the axis computation has a divide by $2$, but since it's being  normalized anyway, can skip the multiplication by a constant

- $  R' = \frac R {||R||} $  

or

- $\theta = 2 \cos^{-1} ( \frac { (A \cdot B) ( \cos \frac {b + a} 2  - \cos \frac {b - a} 2 ) +  \cos \frac {b - a} 2  +  \cos \frac {b + a} 2 } 2 ) $



### notes:
 - To apply inverse multiply $a$ by $-1$ 
 - to apply a partial timestep scale $a$ by $dT$





---

# Interpolation

### Given
- A frame $Q$ as $(Q,q)$
- Another frame $P$ and $(P,p)$
- A scalar $D$ from $0$ to $1$ of the orientation between the two frames to compute.

### result

- $Q'$ as as the partial step between $Q$ and $P$ indicated by $D$.

## Operations

- Apply `Rotate A around B` to rotate $P$ around the $Q$ inversed, result with ($T$, $t$);
- Apply `Rotate vector V to V' ` to rotate $T$ around $Q$, result with $T'$
- use `Rotate A around B` to rotate $Q$ around ($T'$,$t \cdot D$).  Result with $Q'$  as $(Q',q')$


# Integration

### given
- $Q$ as a frame 
- $V$ as a velocity to rotate the frame

### result
- $R$ updated frame

### Operations
internal force

- Apply `Rotate vector V to V'` to rotate $V$ around ${Q}$; moves local $V$ to global $V'$
- Apply `Rotate A around B` to rotate ${Q}$ around $(V',v * dT)$

external force
  
- Apply `Rotate A around B` to rotate ${Q}$ around $(V,v * dT)$

